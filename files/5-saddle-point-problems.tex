% !TeX source = ../main.tex

\chapter{Saddle point problems}\label{chap:saddle}

\section{BNB conditions}
In this section we will examine a more general version of the classic problem \eqref{eqn:weak_laxmilgram}, in which the target space of the operator $A$ is allowed to be different from the dual space of the domain.
\begin{remark}
    Recall that a Banach space $W$ is said to be \emph{reflexive} if the natural map from $W$ to $W''$ is an isomorphism (it is, in fact, a canonical isomorphism, since it does not depend on the choice of a basis). This map takes a vector $w$ and associates it to the linear map $J_w$ such that $J_w(f)=f(w)$ for every $f\in W'$. In practice we require this hypothesis so that we can identify $W=W''$.
\end{remark}
Consider a Banach space $V$ and a \emph{reflexive} Banach space $W$. As usual, we have the option to either consider a linear continuous operator $A\in\L(V;W')$ or a bilinear operator $a\in\L(V\times W;\R)$. Given $f\in W'$, the problem we want to solve is: find $u\in V$ such that
\begin{equation}\label{eqn:petrov-galerkin}
    Au=f \ \text{in} \ W', \ \text{or equivalently} \ a(u,w) = \langle f,w \rangle \ \forall w\in W,
\end{equation}
where as usual $a(\cdot,\cdot)$ is defined by $a(u,w):=\langle Au,w \rangle$ for all $u\in V$ and $w\in W$.\par
The first question we must ask ourselves is whether the problem is well-posed. Note that we can not rely on the Lax-Milgram Lemma since here the bilinear operator is $a:V\times W\to \R$, with $V\neq W$ in general.
\begin{definition}\label{eqn:BNB}\marginpar{BNB conditions}
    A problem of the form \eqref{eqn:petrov-galerkin} is said to be \emph{well-posed} (in the sense of Hadamard) if the following two conditions are satisfied:
    \begin{itemize}
        \item for every $f\in W'$, there exists a unique $u\in V$ such that $Au=f$,
        \item $A$ is bounding, i.e. there exists $\alpha>0$ such that $\norm{u}_V \leq \tfrac{1}{\alpha}\norm{Au}_{W'}$ for every $u\in V$.
    \end{itemize}
\end{definition}
The previous two conditions are also known as \emph{BNB (Banach-Ne\v{c}as-Babuska) conditions} in the literature.
\begin{remark}\label{rmk:bnb_conditions}
	The BNB conditions can be summarized by saying that $A$ is required to be:
	\begin{enumerate}
		\item injective,
		\item surjective,
		\item bounding.
	\end{enumerate}
	Note that injectivity is implied by being bounding, since no element in $\ker A$, other than zero, can satisfy the bounding condition.
\end{remark}


\subsection{Open mapping and closed range theorems}
Injectivity, surjectivity and the bounding condition are all different aspects of the \emph{open mapping theorem} and the \emph{closed range theorem}. Let us recall them briefly in a general setting.

\rev{after checking online, "polar" and the $^o$ symbol seem to refer to another kind of subset, below I used only "annihilator", with the $^\perp$ symbol}
\begin{definition}
	Let $X$ be a Banach space. If $Z \subset X$, we define the \emph{annihilator} of $Z$ as the set
	\[
	Z^\perp = \Set{f \in X' \vert \langle f, z \rangle = 0 \, \forall z \in Z}.
	\]
\end{definition}
It is easy to prove that the annihilator of a set is closed.

\begin{definition}
	Let $X$, $Y$ be Banach spaces. Given $A \in \L(X;Y)$, its \emph{adjoint} is the operator $A^T \in \L(Y';X')$ given by
	\[
	\langle A^T y', x \rangle = \langle y', Ax \rangle, \qquad x \in X, y' \in Y'.
	\]
\end{definition}
Recall that, by continuity, the kernel of a linear continuous map is closed. 

\begin{theorem}[Open mapping theorem]\label{thm:open_mapping}
	Let $T: E \to F$ be a surjective linear continuous map between Banach spaces. Then $T$ is an open mapping, i.e., if $U \subset E$ is an open set, then also $T(U)$ is open.
\end{theorem}

\begin{theorem}[Closed range theorem]\label{thm:closed_range}
	aa
	\rev{insert theorem}
\end{theorem}

\subsection{Conditions equivalent to BNB}

\begin{theorem}\label{thm:closed_range2}
    Let $A\in\L(V;W')$ for $V,W$ Banach spaces. Then the following are equivalent:
    \begin{enumerate}
        \item $\im A = \overline{\im A}$,
        \item there exists $\alpha >0$ such that
        \begin{equation}
        	\forall w' \in \im A \, \exists u \in V \, \text{s.t. }\, Au = w' \, \text{and }\, \norm{Au}_{W'} = \norm{w'}_{W'} \ge \alpha\norm{u}_V.
        \end{equation}
    \end{enumerate}
\end{theorem}
\begin{proof}
    \hfill
    \begin{enumerate}[itemindent=25pt]
        \item[1. $\implies$ 2.] By hypothesis, $\im A$ is a closed linear subspace of $W'$, hence it is a Banach space. As a consequence, $A: V \to \im A$ is still a linear continuous operator, moreover it is surjective. Consider now the open unit ball $B_1(0_V)$: by the open mapping theorem, $A(B_1(0_V))\subseteq\im A$ is open in $\im A$. Therefore, being $A(0_V)=0_{W'}$ by linearity, there exists $\gamma>0$ such that $B_\gamma(0_{W'})\subseteq A(B_1(0_V))$.\par
        Pick any $\alpha < \gamma$ and let $w'\in \im A$. If $w' = 0_{W'}$, then it is sufficient to choose $u = 0_V$. Otherwise, $\overline{w'}:=\alpha\frac{w'}{\norm{w'}_{W'}}\in B_\gamma(0_{W'})$, which is contained in the image of $B_1(0_V)$. This in turn implies the existence of a vector $z\in B_1(0_V)$ such that $Az=\overline{w'}$.\par
        From this, we can easily conclude: take $u:=\frac{\norm{w'}_{W'}}{\alpha}z$, so that $Au=w'$. Then we have
        \begin{equation*}
            \norm{u}_V=\frac{\norm{w'}_{W'}}{\alpha}\norm{z}_V\leq\frac{\norm{w'}_{W'}}{\alpha}=\frac{\norm{Au}_{W'}}{\alpha}.
        \end{equation*}
        \item[2. $\implies$ 1.] We need to prove that $\im A$ is closed in $W'$. Take a Cauchy sequence $\{w_n'\}\subset \im A$ and let $w'\in W'$ be its limit. If we prove that $w'\in\im A$ we are done. For every $w_n'$, take $v_n\in V$ such that $Av_n=w_n'$. Then
        \begin{equation*}
            \norm{w_n'-w_m'}_{W'}=\norm{A(v_n-v_m)}_{W'}\geq\alpha\norm{v_n-v_m}_V,
        \end{equation*}
        hence also $\{v_n\}$ is Cauchy in $V$. If we call $v\in V$ the limit, by continuity of $A$ we must have $Av=w'$.
    \end{enumerate}
\end{proof}

It is useful to relate the bounding condition to other equivalent properties.
\begin{proposition}\marginpar{inf-sup condition}\label{prop:inf-sup_condition}
    Let $V, W$ be Banach spaces, with $W$ reflexive. Let $A\in\L(V;W')$. Then the following are equivalent:
    \begin{enumerate}
    	\item $A^T \in \L(W,V')$ is surjective.
    	\item $A$ is injective and $\im A = \overline{\im A}$.
    	\item $A$ is bounding.
    	\item The \emph{inf-sup condition} is satisfied:
		\begin{equation}\label{eqn:infsup_condition}
			\exists \, \alpha>0 \, \text{s.t. } \inf_{u\in V}\sup_{w\in W}\frac{\langle Au,w \rangle}{\norm{u}_V\norm{w}_W} \ge \alpha.
		\end{equation}    	
    \end{enumerate}
\end{proposition}
\begin{proof}
	The equivalence between 3. and 4. follows directly from the definition of operator norm:
	\[
	\norm{Au}_{W'} = \sup_{w\in W}\frac{\langle Au,w \rangle}{\norm{w}_W}.
	\]
	In fact, the inf-sup condition can be rewritten as
	\begin{equation*}
		\inf_{u\in V}\frac{\norm{Au}_{W'}}{\norm{u}_V}\geq\alpha,
	\end{equation*}
	which is clearly equivalent to the bounding condition.
		
	Let us prove that 1. $\iff$ 2. $\iff$ 3.
	\begin{enumerate}[itemindent=25pt]
		\item[1. $\implies$ 3.] Assume that this is not the case; then there exists a sequence $\{v_n\}\subset V$ with $\norm{v_n}_{V}=1$ and $\norm{Av_n}_{W'}<1/n$ for all $n\in\N$. Note that, since $\norm{v_n}_{V}=1$, for every $n$ there exists a functional $v_n'\in V'$ such that $|v_n'(v_n)|\geq 1/2$. Since $A^T$ is surjective, by the open mapping theorem there exists $\gamma > 0$ such that $A^T(B_1(0_W))$ contains $B_\gamma(0_{V'})$. Thus we get a sequence $\{w_n\}\subset B_1(0_W)$ such that $A^Tw_n=\gamma v_n'$ for all $n$. Finally, observe that this implies that
		\begin{equation*}
			|Av_n(w_n)|=|A^Tw_n(v_n)|=\gamma|v_n'(v_n)|\geq\gamma/2,
		\end{equation*}
		hence $\norm{Av_n}_{W'}\geq\gamma/2$, a contradiction.
		\item[3. $\implies$ 2.] This is a consequence of Theorem \ref{thm:closed_range2}, together with Remark \ref{rmk:bnb_conditions}.
		\item[2. $\implies$ 1.] Being $\im A$ closed, the closed range theorem \ref{thm:closed_range} implies that $\im (A^T) = (\ker A)^\perp$. Since $A$ is injective, $(\ker A)^\perp = V$.
	\end{enumerate}
\end{proof}

In a similar fashion, if $V$ is reflexive as well, one can prove an analogous result for $A^T\in\L(W;V')$. Putting things together, we get an interesting consequence.
\begin{theorem}\label{thm:bnb_boundingness}
	Let $V$, $W$ be reflexive Banach spaces, let $A \in \L(V;W')$. Then, $A$ satisfies the BNB conditions \eqref{eqn:BNB} if and only if both $A$ and $A^T$ are bounding, i.e., there exists $\alpha>0$ such that
	\begin{enumerate}
		\item $\norm{Au}_{W'}\geq\alpha\norm{u}_V \quad  \forall u \in V$,
		\item $\norm{A^Tw}_{V'}\geq\alpha\norm{w}_W \quad  \forall w \in W$.
	\end{enumerate}
\end{theorem}

\begin{remark}
	The bounding conditions above can be more suggestively written as
	\begin{align}
		\inf_{u\in V}\sup_{w\in W}\frac{\langle Au,w \rangle}{\norm{u}_V\norm{w}_W}&\geq\alpha \\
		\inf_{w\in W}\sup_{u\in V}\frac{\langle Au,w \rangle}{\norm{u}_V\norm{w}_W}&\geq\alpha.
	\end{align}
	This is true whenever both $V$ and $W$ are reflexive. In particular, when they are Hilbert spaces.
	\rev{In the teacher's notes it is said that this is true in the Hilbert setting, but there seems to be no problem also if both the spaces are Banach+reflexive. Am I wrong?}
\end{remark}

We also deduce that the following implications hold in general:
\begin{align}
	\text{$A$ surjective} &\implies \text{$A^T$ injective} \\
	\text{$A^T$ surjective} &\implies \text{$A$ injective}
\end{align}
Unfortunately, the converses fail, unless we deal with finite dimensional spaces. In fact, in finite dimensions, any subspace is closed by default, hence either $\im A$ and $\im (A^T)$ are closed.

\begin{example}
	To understand what might go wrong in infinite dimensions, consider the inclusion
	\[
	A: H_0^1(\Omega) \hookrightarrow L^2(\Omega) = \bigl(L^2(\Omega)\bigr)',
	\]
	i.e., $Au = u$ for all $u\in H_0^1(\Omega)$. Clearly, $\ker(A)=\{0\}$, but $A$ is not surjective: in fact, functions with a jump discontinuity can be in $L^2$ but not in $H^1_0$. Moreover, $\im A = H_0^1(\Omega)$ is dense in $L^2(\Omega)$ because $H_0^1$ is the closure of $C_0^\infty$, which is dense in $L^2$. Hence, $\im A\neq\overline{\im A}$.
	
	If we were in finite dimensions, $\ker A = \{0\}$ would be sufficient for $A$ to be invertible. However, let us look at this example from another point of view. The adjoint operator $A^T: L^2(\Omega) \to \bigl(H_0^1(\Omega)\bigr)'$ satisfies
	\[
	(A^T q)(u) = \int_{\Omega} uq \qquad \forall q \in L^2(\Omega).
	\]
	If $(A^T q)(u) = 0$ for every $u \in H_0^1(\Omega)$, then $q = 0$, thanks to standard results (fundamental lemma of calculus of variations). Hence, $\ker (A^T)$ is trivial. In finite dimensions, having $\ker (A^T) = \{0\}$ would imply that $A$ is surjective, which is not the case.
\end{example}

Finally, we conclude this section with the following result, which we do not prove.
\begin{theorem}
    Using the notations above, if we assume that $V=W$ and that $A$ is $\alpha$-elliptic, then ``Lax-Milgram implies BNB'', i.e. if the hypothesis of Lemma \ref{lemma:lax-milgram} are satisfied, then \eqref{eqn:BNB} holds.
\end{theorem}


\section{Ceà's Lemma for Petrov-Galerkin method}
We consider the following slightly less general setting for Petrov-Galerkin: let $V,Q$ be Hilbert spaces, let $V_h\subset V$ and $Q_h\subset Q$ be some subspaces of finite dimension, and let $A\in\L(V;Q')$ be a linear continuous operator and let and $A_h\in\L(V_h;Q_h')$ be its discretization. It will prove useful in the following to introduce two \emph{projection} operators from the whole space to the discrete space $\Pi_h:V\to V_h$ and $P_h:Q\to Q_h$ defined as
\begin{equation*}
    \langle\Pi_h v,v_h\rangle=\langle v,v_h\rangle \ \forall\, v\in V,
\end{equation*}
and analogously for $P_h$.\par
Consider the discrete version of problem \eqref{eqn:petrov-galerkin}: given $f\in V'$, find $u_h\in V_h$ such that
\begin{equation}\label{eqn:discrete petrov-galerkin}\marginpar{Discrete Petrov-Galerkin}
    \langle Au_h, q_h \rangle= \langle f,q_h \rangle \ \forall q_h\in Q_h.
\end{equation}
\begin{remark}
    The following observation will be useful later on. Note that if $u$ is a solution to the continuous problem, then $\langle Au,q_h\rangle=\langle f,q_h\rangle$ for all $q_h\in Q_h\subset Q$. At the same time, given a solution $u_h$ to the discrete problem, we have $\langle Au_h,q_h\rangle=\langle f,q_h\rangle$ for all $q_h\in Q_h$. Subtracting the two gives us
    \begin{equation}\label{eqn:petrov-galerkin orthogonality}
        \langle A(u-u_h),q_h \rangle=0 
    \end{equation}
    for all $q_h\in Q_h$.
\end{remark}
As it has been discussed in the previous section, the solution exists and is unique if and only if the BNB conditions hold: in this setting we call them \emph{discrete infsup conditions} (or \emph{discrete BNB conditions}) to distinguish them from the infsup conditions for the continuous problem. Their formulation is however exactly the same:
\begin{align}\label{eqn:discrete infsup}\marginpar{Discrete infsup}
    \inf_{u_h\in V_h}\sup_{q_h\in Q_h}\frac{\langle Au_h,q_h \rangle}{\norm{u_h}_V\norm{q_h}_{Q_h}}&\geq\alpha_h>0 \\
    \inf_{q_h\in Q_h}\sup_{u_h\in V_h}\frac{\langle Au_h,q_h \rangle}{\norm{u_h}_V\norm{q_h}_{Q_h}}&\geq\alpha_h>0.
\end{align}
\begin{remark}
    We wrote $\alpha_h$ to stress the fact that this constant is generally different to the one in the continuous infsup, but $\alpha_h$ \emph{does not} depend on $h$.
\end{remark}
\begin{lemma}[Ceà's lemma for Petrov-Galerkin]
    Assume that the infsup conditions hold for both the continuous and the discrete problem. Let $u\in V$ be the solution to \eqref{eqn:petrov-galerkin} and let $u_h\in V_h$ be the solution to \eqref{eqn:discrete petrov-galerkin}. Then
    \begin{equation*}
        \norm{u-u_h}_V\leq\left( 1+\frac{\norm{A}}{\alpha_h} \right)\inf_{v_h\in V_h}\norm{u-v_h}_V.
    \end{equation*}
\end{lemma}
\begin{proof}
    From the discrete infsup, we have
    \begin{equation*}
        \alpha_h\leq \inf_{v_h\in V_h}\sup_{q_h\in Q_h}\frac{\langle Av_h,q_h \rangle}{\norm{v_h}_V\norm{q_h}_{Q_h}}= \inf_{v_h\in V_h}\frac{\norm{Av_h}_{*,Q_h}}{\norm{v_h}_V},
    \end{equation*}
    where $\norm{\cdot}_{*,Q_h}$ is the operator norm on the space $Q_h$ (i.e. the last equality follows by the definition of $\norm{\cdot}_{*,Q_h}$). From this follows that
    \begin{equation*}
        \frac{1}{\alpha_h}\norm{Av_h}_{*,Q_h}\geq\norm{v_h}_V \quad \forall\, v_h\in V_h.
    \end{equation*}
    In particular, it is also true that
    \begin{equation*}
        \frac{1}{\alpha_h}\norm{A(u_h-v_h)}_{*,Q_h}\geq\norm{u_h-v_h}_V \quad \forall\, v_h\in V_h,
    \end{equation*}
    since clearly $u_h-v_h\in V_h$. Now, we start doing some inequalities:
    \begin{align*}
        \norm{u-u_h}_V&\leq\norm{u-v_h}_V+\norm{v_h-u_h}_V\\
        &\leq\norm{u-v_h}_V+\frac{1}{\alpha_h}\norm{A(v_h-u_h)}_{*,Q_h}
    \end{align*}
    for all $v_h\in V_h$. At this point, to conclude it is enough to observe that
    \begin{align*}
        \norm{A(v_h-u_h)}_{*,Q_h}&=\sup_{q_h\in Q_h}\frac{\langle A(v_h-u_h),q_h \rangle}{\norm{q_h}_{Q_h}}\\
        &=\sup_{q_h\in Q_h}\frac{\langle A(v_h-u),q_h \rangle + \langle A(u-u_h),q_h\rangle}{\norm{q_h}_{Q_h}}\\
        &=\sup_{q_h\in Q_h}\frac{\langle A(v_h-u),q_h \rangle}{\norm{q_h}_{Q_h}}\\
        &=\norm{A(v_h-u)}_{*,Q_h},
    \end{align*}
    where we used \eqref{eqn:petrov-galerkin orthogonality} for the third equality. Plugging this into the inequality we obtained previously yields
    \begin{align*}
        \norm{u-u_h}_V&\leq\norm{u-v_h}_V+\frac{1}{\alpha_h}\norm{A(v_h-u)}_{*,Q_h}\\
        &\leq\left( 1+\frac{\norm{A}}{\alpha_h} \right)\norm{v_h-u}_V,
    \end{align*}
    for all $v_h\in V_h$. Taking the inf on $V_h$ gives the wanted inequality.
\end{proof}


\section{Mixed Problems}
Saddle point problems are a class of optimization problems characterized by the presence of both primal and dual variables, leading to solutions that are critical points of a Lagrangian function. These problems are common in various applications such as fluid dynamics, structural mechanics, and optimization.\par
The setting is the following: take  two Hilbert spaces $V,Q$ and two linear continuous operators
\begin{equation*}
    A:V\to V', \quad B:V\to Q'.
\end{equation*}
Take also $f\in V'$ and $g\in Q'$. The problem we aim to solve is to find $(u, p)\in V\times Q$ such that:
\begin{equation}\label{eqn:general mixed prb} \marginpar{General mixed problem}
    \begin{aligned}
        Au + B^Tp &= f \quad \text{in} \ V', \\
        Bu &= g  \quad \text{in} \ Q'.
    \end{aligned}
\end{equation}

\begin{remark}
    Assume that $g$ is in the image of $B$. Then there exists $u_g\in V$ such that $Bu_g=g$. Call $Z:=\ker{B}$; then the solution $u\in V$ can be written as $u=u_0 + u_g$ for some $u_0\in Z$. Substituting in \eqref{eqn:general mixed prb} we obtain:
    \begin{equation}\label{eqn:u0 prb}
        \begin{aligned}
            Au_0 + B^Tp &= f - Au_g=:\Tilde{f} , \\
            Bu_0 &= 0,
        \end{aligned}
    \end{equation}
    and the second equation is satisfied automatically by construction.
\end{remark}

By the previous Remark, we can therefore restrict ourselves to the study of the so-called ``$u_0$ problem'', i.e. finding $u_0\in Z$ such that
\begin{equation*}
    \langle Au_0,v_0 \rangle + \langle B^Tp,v_0 \rangle = \langle \Tilde{f},v_0 \rangle \quad \forall \ v_0\in Z
\end{equation*}
(this is just the first \eqref{eqn:u0 prb} written in terms of scalar products). Since $v_0\in Z$, the second term of the left-hand side vanishes, and thus we are left with the equation
\begin{equation*}
    \langle Au_0,v_0 \rangle = \langle \Tilde{f},v_0 \rangle \quad \forall \ v_0\in Z.
\end{equation*}
As we have said multiple times, the solution $u_0\in Z$ to this problem exists and is unique if and only if the BNB conditions hold in $Z$. Explicitly, we ask that there exists an $\alpha>0$ such that
\begin{align}\label{eqn:ell-ker conditions}\marginpar{Ell-ker conditions}
    \inf_{v_0\in Z}\sup_{u_0\in Z}\frac{\langle Au_0,v_0 \rangle}{\norm{u_0}_V\norm{v_0}_V}&\geq\alpha \\
    \inf_{u_0\in Z}\sup_{v_0\in Z}\frac{\langle Au_0,v_0 \rangle}{\norm{u_0}_V\norm{v_0}_V}&\geq\alpha.
\end{align}
In this setting, the BNB conditions are also called \emph{ell-ker conditions} (for ellipticity in the kernel).\par
If we manage to solve the $u_0$-problem, the solution to the original problem can be found by solving the \emph{$p$-problem}: find $p\in Q$ such that
\begin{align*}
    \langle B^Tp,v\rangle &= -\langle Au,v\rangle + \langle f,v\rangle\\
    &= -\langle Au_0,v\rangle + \langle\Tilde{f},v\rangle \quad \forall v\in V.
\end{align*}
Once again, existence and uniqueness of the solution are equivalent to the following infsup condition:
\begin{equation}\label{eqn:infsup for B}\marginpar{Infsup condition for $B$}
    \inf_{p\in Q}\sup_{v\in V}\frac{\langle Bv,p \rangle}{\norm{v}_V\norm{p}_Q}\geq\beta>0
\end{equation}
In conclusion, if conditions \eqref{eqn:ell-ker conditions} and \eqref{eqn:infsup for B} hold, then there is a unique solution $(u,p)\in V\times Q$ for the mixed problem \eqref{eqn:general mixed prb}.\par
Now we want to investigate whether we can estimate the norm of the solution in terms of $f,g$ and the infsup constants $\alpha$ and $\beta$. First observe that, from \eqref{eqn:ell-ker conditions} and the definition of $\Tilde{f}$, it follows immediately that
\begin{equation*}
    \norm{u_0}_V\leq\frac{1}{\alpha}\norm{\Tilde{f}}_{V'}\leq\frac{1}{\alpha}\left( 
\norm{f}_{V'}+\norm{A}_*\norm{u_g}_V \right).
\end{equation*}
Moreover, from \eqref{eqn:infsup for B} and the definition of $u_g$ we have
\begin{equation*}
    \norm{g}_{Q'}=\norm{Bu_g}_{Q'}\geq\beta\norm{u_g}_V.
\end{equation*}
Hence 
\begin{align*}
    \norm{u}_V &= \norm{u_0+u_g}_V\\
    &\leq\frac{1}{\alpha}\left(\norm{f}_{V'}+\norm{A}_*\norm{u_g}_V \right)+\frac{1}{\beta}\norm{g}_{Q'}\\
    &\leq \frac{1}{\alpha}\norm{f}_{V'}+\left( \frac{\alpha+\norm{A}_*}{\alpha\beta}\right)\norm{g}_{Q'}.
\end{align*}
An analogous computation can be made for $p,$ showing that
\begin{equation*}
    \norm{p}_Q\leq\frac{\norm{A}_*+1}{\beta}\norm{f}_{V'}+\frac{\norm{A}_*}{\beta}\left(\frac{\norm{A}_*+\alpha}{\alpha\beta}\right)\norm{g}_{Q'}.
\end{equation*}
\begin{remark}
    Call $\mathbb{V}:=V\times Q$ and
    \begin{equation*}
        \mathbb{A}:=\begin{bmatrix}
            A & B^T\\
            B & 0
        \end{bmatrix}\in\L(\mathbb{V};\mathbb{V}').
    \end{equation*}
    The problem \eqref{eqn:general mixed prb} can clearly be stated in terms of a singular variable $\psi:=(u,p)\in\mathbb{V}$ and the linear operator $\mathbb{A}$. It can be proved (but it is not easy) that the conditions \eqref{eqn:ell-ker conditions}, \eqref{eqn:infsup for B} are equivalent to the existence of a constant $\overline{\alpha}>0$ such that
    \begin{align}
        \inf_{\psi\in \mathbb{V}}\sup_{\theta\in \mathbb{V}}\frac{\langle A\psi,\theta \rangle}{\norm{\psi}_\mathbb{V}\norm{\theta}_\mathbb{V}}&\geq\overline{\alpha} \\
        \inf_{\theta\in \mathbb{V}}\sup_{\psi\in \mathbb{V}}\frac{\langle A\psi,\theta \rangle}{\norm{\psi}_\mathbb{V}\norm{\theta}_\mathbb{V}}&\geq\overline{\alpha}.
    \end{align}
    This is known in the literature as \emph{``Brezzi $\iff$ Babuska''}.
\end{remark}

\subsection{Some considerations on discrete mixed problems}
The aim of this paragraph is to give an idea as to why mixed problem are difficult. As we will see shortly, the main reason is that the ``goodness'' of the approximation for the velocity does not only depend on the approximation space $V_h$, but also depends on the choice of $Q_h$ (and the same can be said for the approximation of the pressure).\par
The main idea here is to proceed as usual, as if we wanted to prove Ceà's lemma. But first we introduce the following object, which is basically a discrete version of $\ker(B)$:
\begin{equation*}
    Z_h:=\{ v_h\in V_h \mid \langle Bv_h,q_h\rangle=0 \ \forall\,q_h\in Q_h \}.
\end{equation*}
\begin{remark}
    In general, it is not true that $Z_h\subset Z$. For example, we can consider $B=\dvg$ in dimension 1 (i.e. $B$ is the derivative operator), so that $Z$ is the set of (a.e.) constant function. However, if we consider $Q_h$ to be the set of piece-wise constant functions, then
    \begin{equation*}
        \langle Bu,q_h\rangle=0 \ \forall\, q_h\in Q_h \ \iff \ \int_T u'=0 \ \forall \, T \ \text{triangle in the triangulation}.
    \end{equation*}
    Clearly the set of functions with 0 derivative on each element of the triangulation is not contained in the set of constant functions, i.e. $Z_h\not\subset Z$.\par
    Also note that in general it is also not true that $Z\subset Z_h$.
\end{remark}
With the usual notations, we let $(u,p)$ be the solution to \eqref{eqn:general mixed prb} and let $(u_h,p_h)$ be the approximated solution. Then we can write
\begin{align*}
    \langle Au,v_h\rangle+\langle Bv_h,p\rangle&=\langle f,v_h\rangle \quad \forall\,v_h\in V_h\\
    \langle Au_h,v_h\rangle+\langle Bv_h,p_h\rangle&=\langle f,v_h\rangle \quad \forall\,v_h\in V_h,\\
\end{align*}
Subtracting the two gives
\begin{equation*}
    \langle A(u-u_h),v_h\rangle=-\langle Bv_h,p-p_h\rangle \quad \forall\, v_h\in V_h;
\end{equation*}
now observe that if we restrict ourselves to $v_h\in Z_h\subset V_h$ then $\langle Bv_h,p-p_h\rangle=\langle Bv_h,p-q_h\rangle$ for all $q_h\in Q_h$, so the previous equation can be rephrased as
\begin{equation}\label{eqn:mixed prb orthogonality}
    \langle A(u-u_h),v_h\rangle=-\langle Bv_h,p-q_h\rangle \quad \forall\, v_h\in Z_h \ \text{and} \ \forall\,q_h\in Q_h.
\end{equation}
As usual we can employ the discrete infsup condition \eqref{eqn:discrete infsup} to write
\begin{align*}
    \norm{u-u_h}_V&\leq\norm{u-v_h}_V+\norm{v_h-u_h}_V\\
    &\leq\norm{u-v_h}_V +\frac{1}{\alpha_h}\norm{A(v_h-u_h)}_{*,Z_h}
\end{align*}
for all $v_h\in Z_h$. Finally, we use \eqref{eqn:mixed prb orthogonality} in the following chain of (in)equalities:
\begin{align*}
    \norm{A(v_h-u_h)}_{*,Z_h} &= \sup_{z_h\in Z_h}\frac{\langle A(v_h-u_h),z_h\rangle}{\norm{z_h}_{V_h}}\\
    &= \sup_{z_h\in Z_h}\frac{\langle A(v_h-u),z_h\rangle + \langle A(u-u_h),z_h\rangle}{\norm{z_h}_{V_h}}\\
    &= \sup_{z_h\in Z_h}\frac{\langle A(v_h-u),z_h\rangle - \langle Bz_h,p-q_h\rangle}{\norm{z_h}_{V_h}}\\
    &= \norm{A(v_h-u)}_{*,Z_h}+\norm{B^T(q_h-p)}_{*,Z_h}\\
    &\leq \norm{A}_*\norm{v_h-u}_V+\norm{B}_*\norm{q_h-p}
\end{align*}
for all $v_h\in Z_h$ and all $q_h\in Q_h$. Taking the inf on $Z_h$ and $Q_h$ gives a similar result to the classic Ceà's lemma for the velocity:
\begin{equation*}
    \norm{u-u_h}_V\leq \left( 1+\frac{\norm{A}_*}{\alpha_h}\right)\inf_{v_h\in Z_h}\norm{u-v_h}_V+\frac{\norm{B}_*}{\alpha_h}\inf_{q_h\in Q_h}\norm{p-q_h};
\end{equation*}
similarly, we can obtain the following inequality for the pressure:
\begin{equation*}
    \norm{p-p_h}_Q\leq \left( 1+\frac{\norm{B}_*}{\beta_h}\right)\inf_{q_h\in Q_h}\norm{p-q_h}_Q+\frac{\norm{A}_*}{\beta_h}\inf_{v_h\in Z_h}\norm{u-v_h}.
\end{equation*}



\subsection{An example: the mixed Laplacian}
Another example is the mixed formulation of the Laplacian:
\begin{equation}
    \begin{aligned}
        a(u, v) + b(v, p) &= f(v) \quad \forall v \in V, \\
        b(u, q) &= g(q) \quad \forall q \in Q.
    \end{aligned}
\end{equation}

\rev{complete this}